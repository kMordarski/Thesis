% !TEX TS-program = pdflatexmk

% Style for a MSc paper at Warsaw School of Economics
% Michał Ramsza
% Friday, December 14, 2012

% --- document class and other global stuff ---------------------------
\documentclass[english, twoside, 12pt, a4paper]{article}

% --- packages --------------------------------------------------------
\usepackage{textcomp}
\usepackage{times}
\usepackage{amsmath}
\usepackage{amsfonts}
\usepackage{amssymb}
\usepackage{amsthm}
\usepackage[T1]{fontenc}
\usepackage[utf8]{inputenc}
\usepackage{graphicx}
\usepackage{xcolor}
\usepackage{enumitem}
\usepackage[english]{babel}
\usepackage[centering, left=3.5cm, right=2.5cm, textheight=24cm]{geometry}
\usepackage{listings}
\usepackage[]{algorithm2e}

% --- packages for citations ------------------------------------------
\usepackage{natbib}
\AtBeginDocument{\renewcommand{\harvardand}{and}}

% --- package for automatic insertion of R code -----------------------
\usepackage{listings}
\lstset{language=R,%
   numbers=left,%
   tabsize=3,%
   numberstyle=\footnotesize,%
   basicstyle=\ttfamily \footnotesize \color{black},%
   escapeinside={(*@}{@*)}}

% --- support for links -----------------------------------------------	
\usepackage{url}
\usepackage{hyperref}
\hypersetup{colorlinks=true,
            linkcolor=black,
            citecolor=darkgray,
            urlcolor=darkgray,
            pagecolor=darkgray}

% --- support for large tables and other stuff ------------------------	
\usepackage{longtable}
%\usepackage{subfigure} % this package will not work with subcaption package
\usepackage{float}
\usepackage{caption}
\usepackage{subcaption}
\usepackage{wrapfig}
\usepackage{pdflscape} % relevant for wide tables (rotating pages)

% --- support for game theory ------------------------------------------
\usepackage{sgame}

% --- support for no widows --------------------------------------------
\usepackage[defaultlines=4,all]{nowidow}

% --- quotation for polish language \enquote{}
\usepackage[autostyle]{csquotes}
\DeclareQuoteAlias{dutch}{polish}

% --- definitions for environments -------------------------------------
\theoremstyle{definition}
    \newtheorem{condition}{Assumption}
    \newtheorem{example}{Example}      

\theoremstyle{plain}
    \newtheorem{definition}{Definition}    
    \newtheorem{proposition}{Proposition}
    \newtheorem{theorem}{Theorem}
    \newtheorem{cor}{Corollary}

\theoremstyle{remark}
    \newtheorem{remark}{Remark}

% --- other settings --------------------------------------------------
\linespread{1.5}
\frenchspacing
\sloppy
\allowdisplaybreaks[4]
\raggedbottom
\clubpenalty=10000
\widowpenalty=10000

% --- only if required ------------------------------------------------
\AtBeginDocument{\renewcommand*{\figurename}{Figure}}
\AtBeginDocument{\renewcommand*{\tablename}{Table}}

% --- changing definition of footnote ---------------------------------
\makeatletter
\renewcommand\footnotesize{%
   \@setfontsize\footnotesize\@ixpt{10}%
   \abovedisplayskip 8\p@ \@plus2\p@ \@minus4\p@
   \abovedisplayshortskip \z@ \@plus\p@
   \belowdisplayshortskip 4\p@ \@plus2\p@ \@minus2\p@
   \def\@listi{\leftmargin\leftmargini
               \topsep 4\p@ \@plus2\p@ \@minus2\p@
               \parsep 2\p@ \@plus\p@ \@minus\p@
               \itemsep \parsep}%
   \belowdisplayskip \abovedisplayskip
}
\makeatother

\newcommand{\todo}[1]{\noindent{\color{red}>>~#1}}

% ---------------------------------------------------------------------
\begin{document}

% --- strona tytulowa -------------------------------------------------
\begin{titlepage}
\centering

\includegraphics[width=0.66\textwidth]{logo.JPG}

\vspace*{0.5cm}
Studium licencjackie\\
\begin{flushleft}
Kierunek: Metody Ilościowe w Ekonomii i Systemy Informacyjne\\
%Specjalność: <specjalność> % w przypadku braku należy pominać
Forma studiów: Stacjonarne
\end{flushleft}

\vspace*{.5cm}
\rule{0cm}{1cm}\hfill
\begin{minipage}{9cm}
Imie i nazwisko autora: Kacper Mordarski\\
Nr albumu: 101247
\end{minipage}

\vspace*{1cm}
\begin{minipage}{12cm}
\centering
\Large
\textbf{<tytuł>}
\end{minipage}

\vspace*{2cm}
\rule{0cm}{1cm}\hfill
\begin{minipage}{9cm}
Praca licencjacka napisana\\
w Katedrze Matematyki i Ekonomii Matematycznej\\
pod kierunkiem naukowym\\
dr hab. Michała Ramszy
\end{minipage}

\vfill
Warszawa 2022
\end{titlepage}

\rule{1ex}{0ex}\clearpage

% --- table of contents -----------------------------------------------
\cleardoublepage
\tableofcontents

% --- chapter ---------------------------------------------------------
\cleardoublepage
\section{Introduction}
\subsection{General Introduction}

This paper aims to replicate, to some degree, the seminal paper of \cite{santos2005scale} on the emergence of cooperation. The secondary goal of this research is to provide the publicly available code allowing replication of the actual results.

\todo{To co jest poniżej to do poprawki. Trzeba tutaj napisać taki standardowy wstęp machany rękami. Koniecznie trzeba się pozbyć ``The paper'', to jednak jest nie do przyjęcia.}

either prove or disprove the research paper with title \textbf{"Scale-Free Networks Provide a Unifying Framework for the Emergence of Cooperation"}  written by F.C. Santos and J. M. Pacheco and published in Physical Review Letters nr 95 (later on reffered to as "The Paper"). Regardless of the outcome this work is going to be useful because the above mentioned authors did not provide the code that was used to conduct the necessary simulations. Therefore goal is not only to disprove the above mantioned paper but also to provide readers with clear and easy to follow code. 
 
 \subsection{Description of The Paper}

The Paper was published in \enquote{Physical Review Letters} on the 26th of August 2005 and, according to Google Scholar, has been since cited over 1600 times. It clearly shows the magnitude of the said paper and its groundbreaking character. This paper presents the results of simulations conducted by the authors and their implications for evolutionary game theory.
 
The crucial thing to understand is that the authors of The Paper changed the approach to modeling such games by applying Scale-Free Networks of Contacts (later on referred to as "SF NOCs"). Its innovativeness lies in the never used before degree distribution of said graph. Before being used graphs had a degree distribution with a single peak. It means that every "player" could interact only with a fixed number of other "players". SF NOCs are said to 
 perform better at modeling the actual, existing societies and networks. It complies with the rules of growth and preferential attachment (rich gets richer). When we analyze some actual networks, for example, the Twitter network, we observe that those with a bigger count of "followers" are more likely to gain new ones than accounts 
 with a low count of followers. 
 
One of the goals of The Paper was to compare the results of simulations on different kinds of graphs. According to The Paper, players that occupy vertices of SF NOC are much more likely to cooperate than on any other graph. Those results came up both in the Snowdrift game (later on referred to as SG) and Prisoners Dilemma game (later on referred to as PD).  

\todo{Tutaj opisać jaka jest struktura całej pracy, w sensie w tym rozdziale to jest to, w tamtym rozdziale to jest tamto. Robimy to na końcu.}




% --- chapter ---------------------------------------------------------
\clearpage
\section{Elements of the noncooperative game theory}

\todo{Tutaj trzeba opisać wszystkie koncepcje matematyczne związane z teorią gier. Wydaje się, że to co jest nam potrzebne to definicja gry w postaci normalnej i definicja równowagi Nasha. Na koniec zamieścić przykłady pojedynczych gier (ale niekoniecznie PD i SD. Poza tym warto chyba jednak opisać jakiś ogólny framework, w którym gra jest populacyjna, co w takiej sytuacji jest mierzone i jak to ewoluuje w czasie. Myślę, że tutaj mogę Panu pomóc. Dodałem tutaj dwa cytowania typowe dla modelu: \cite{vega1996evolution,weibull1997evolutionary,ramsza2010elementy}. Typowe cytowanie do teorii gier to \cite{fudenberg1991game,gibbons1992game}.}

The noncooperative game theory is the basis for this dissertation. For a game to be noncooperative means that all of the game participants (players)
act in their own best interest and therefore compete with each other. Moreover they need to be economically rational --- choose a strategy based on its optimality.
Games in this paper will be presented as normal-form games. For the normal-form game to be described, it requires a couple of elements. First of all, 
let us define a set of players $i \in I$, where $I$ is a finite set of natural numbers (i. e. $I = {1,2,...,N}$). 
For each player $i$, we define the pure strategy space $S_i$ consisting of $k$ pure strategies. The sequence of strategies --- $(s_1, ..., s_k) \in \prod_{i \in N} S_i$ is called the strategy profile of a player. 
Lastly we define function (denoted as $u_i(s)$). This function maps the von Neumann-Morgenstern utility (payoff) for each profile of strategies for a given player.
Methematically speaking we can denote this function as $u_i: \prod_{i \in N}S_i \rightarrow \mathbb{R}$. The last two assumption of 
a normal-form game is that the structure of the game is perfectly known --- each player knows all of the possible strategies and utility functions in the given round of a game and 
that players have no knowledge of other player choices. It means that each time they select their strategy simultaneously and thus
cannot respond to the actions of other players. 

   \cite{fudenberg1991game} 



We are going to be discussing noncooperative games played by two players with pure strategies.


% --- chapter ---------------------------------------------------------
\clearpage
\section{Elements of the graph theory}

\todo{Tutaj musimy opisać wybrane elementy teorii grafów. Myślę, że pojęcie grafu nieskierowanego oraz wybrane grafy losowe, razem z cytowaniami. Tutaj można załączyć przykłady zarówno grafów losowych ale również małych grafów losowych tak aby było jasne jaka jest struktura tych grafów. Dodać oczywiście odpowiednie cytowania.}
40,5

% --- chapter ---------------------------------------------------------
\clearpage
\section{Cooperation among the players}

\todo{W tym miejscu w każdym rozdziale trzeba wpisać krótkie streszczenie co jest w danym rozdziale.}

\subsection{A brief description of Snowdrift and Prisoners Dilemma games}

\paragraph{Prisoners Dilemma Game.} The Prisoners Dilemma game is a widely known problem in game theory and decision analysis. It shows situations in which the outcome is not optimal, even though players act in their own best interest. In the scope of our analysis, it is essential to note that in the PD game, the best strategy is to defect, regardless of the opponent's choice. The game is parameterized as follows:
\begin{center}
\begin{game}{2}{2}
  & $C$    & $D$    \\
$C$ & $R,R$ & $S,T$  \\
$D$ & $T,S$ & $P,P$
\end{game}
\end{center}
where the values are given as follows:
\[
\begin{aligned}
T &= b > 1, & R &= 1, \\
P &= 0, & S &= 0, \\
1 &< b \le 2. 
\end{aligned}
\]
We see that the above restrictions order the parameters as follows:
\[
T > R > P = S .
\]

\paragraph{Snowdrift Game.} The Snowdrift game represents a metaphor for cooperative interactions between players. Contrary to the PD game, the Snowdrift game stimulates cooperative behavior amongst players. It doesn't make the deflection strategy inapplicable, but the game's payoffs encourage cooperative behavior more than the payoffs of the PD game. The optimal strategy is to cooperate when the other defects and to defect when the other cooperates. The game is parametrize as follows:
\begin{center}
  \begin{game}{2}{2}
    & $C$    & $D$    \\
  $C$ & $R,R$ & $T,S$  \\
  $D$ & $S,T$ & $P,P$
  \end{game}
  \end{center}
where the parameters' values are:
\[
  \begin{aligned}
  T &= \beta > 1 , &  R &= \beta - \frac{1}{2} ,\\
  S &= 1 - \beta , &  P &= 0. 
  \end{aligned}
\]
We see that the above restrictions order the parameters as follows:
\[
 T > R > P > S .
\]

\paragraph{Nash equilibria.}

\todo{Tutaj analiza równowag Nasha w opisanych powyżej grach w one-shot games. Pokazać, że jest albo nie mam kooperacji. Tutaj również wprowadzamy grę na grafach losowych, a więc musimy opisać na jakich grafach, itd. Tutaj również opis uczenia się ale w sensie algorytmu matematycznego nie implementacji.}

\subsection{Replicator dynamics}

%
%\subsection{Replicatory dynamics}
%
%In our analysis we consider replicator to be a strategy in a game. The general idea is that replicators compete for dominance throughout the population. Payoffs of
%their strategies represent their "fitness". It is important to note that each player can alter their strategy through inheritance. The attempt of inheritance occurs 
%whenever one of the sites is updated. For the sake of an example, let us say that the site that was just updated is site x. The procedure is as follows:
%
%\begin{enumerate}
%  \item The site x is updated.
%  \item A neighbor y is drawn at random among all k$_{x}$ neighbors
%  \item if cumulative payoffs of y (P$_{y}$) are greater than cumulative payoffs of x (P$_{x}$), the chosen neighbor takes over site x with probability (P$_{i}$) given below.
%\end{enumerate}
%
%
%\begin{center}
%
%\[
%  P_{i} = \frac{(P_{y} - P_{x})}{Dk_{>}}
%  \]
%\end{center}
%Where P$_{i}$ is the probability of the chosen neighbor taking over the site x, P$_{y}$ is a cumulated payoff of strategies y, P$_{x}$ is a cumulated payoff 
%of strategies x, k$_{>}$ is the largest between k$_{y}$ and k$_{x}$ (k$_{y}$ is a number of neighbors with a strategy y, k$_{x}$ is a number of neighbors with a 
%strategy x), D depends on the game (it is equal to either T-S for PD or T-P for SG).

\begin{enumerate}
  \item The site x is updated.
  \item A neighbor y is drawn at random among all k$_{x}$ neighbors
  \item if cumulative payoffs of y (P$_{y}$) are greater than cumulative payoffs of x (P$_{x}$), the chosen neighbor takes over site x with probability (P$_{i}$) given below.
\end{enumerate}

% --- chapter ---------------------------------------------------------
\clearpage
\section{Algorithms and simulations}

This chapter contains a description of the algorithm used in the simulation.

\todo{To co powyżej do rozszerzenia. Generalnie ten rozdział ma zawierać implementację koncepcji matematycznej z poprzedniego rozdziału.}

%\section{Algorithm}
%\subsection{Introduction to the algorithm}




In this section of the dissertation, I'm going to describe the key elements of the algorithm used to replicate the results obtained in The Paper. 
% All of the following code is written in the julia language
My understanding of the mechanisms on which the algorithms are based is limited to the rather vague and unclear description in The Paper. 

\subsection{Functions}

In order to perform necessary calculations I had to define the following functions:

\begin{enumerate}
  \item \lstinline+Transform+ --- This function is used to map strategies onto a vector of arrays. As an input this function takes one of the edges of a SF NOC,
   as an output it returns a vector of length two (two vertices connected with an edge).
  \item Strat --- This function is used to map previously distributed strategies onto a vector of edges. As an input it takes an edge and as an output
   it returnes a vector of length two (two strategies previously attributed to the vertices).
  \item Games --- This function is used to evaluate the results of games played between players (vertices connected with an edge). As an input this function
   takes a vector of edges, vector of strategies and a vector of accumulated payoffs. It returns an adjusted vector of accumulated payoffs.
  \item CheckStrat --- This function fulfills a number of tasks. It takes as an input a randomly chosen vertex, vector of strategies, vector of accumulated
  payoffs, and the SF NOC. It identifies all neighbors of the previously mentioned vertex, then shuffles them, and then looks for a neighbor
  with a different strategy. Then it proceeds to change the strategy of the vertex with lower accumulated payoffs. The probability of the transition 
  is described in section 1.5. 
\end{enumerate}

\subsection{Description of the algorithm}

The main algorithm is the fundamental element of this dissertation. It consists of 3 nested loops executing instructions necessary to conduct simulations, on which my research is based. I will be describing those loops in an inside-out order. 

The first loop is responsible for evaluating the results of the games, potentially changing the strategies of players, and keeping track of the proportion of the strategies used by players. To run properly, it requires previously set parametrization and a set of edges of the Barabasi-Albert graph. As a result, it produces a vector of length 100 in which elements are proportions of coop/def strategies, measured after each generation (1001--1100) in the population of players. This loop is repeated 2 000 times, which gives us a total of 2~200~000 generations for one Barabasi-Albert graph parametrization for a given game.

The loop itself operates as follows:

\begin{enumerate}[noitemsep]
  \item Each pair of connected players (vertices of a graph) engage in a single round of a given game. It means that function GAMES is applied throughout the entire array of edges, adjusting their accumulated payoffs.
  \item One player is chosen randomly.
  \item An attempt to change strategy is being embarked on. It indicates that the function CHECKSTRAT is applied to the player chosen in the previous step.
  \item If the current iteration is higher than 1000, it calculates the share of cooperators in the entire population and then passes it onto a corresponding value of a vector TRACK.
  \item After 1~100 iterations (generations), it ends, and then the next iteration of the outside loop is triggered.
\end{enumerate}

The outer loop to the one described above is responsible for \enquote{resetting} the Barabasi-Albert graph. Since separate simulations are supposed to be carried out on a randomly generated SF NOC (but with the same parametrization), we need to conduct 100 of them (for each payoff parametrization); this loop is an indispensable element of the algorithm. The crucial fact to note is that this loop is also responsible for creating an object named ARRPATHS --- vector of length 100, used to store the results of the simulations.

The procedure followed by this loop is as shown below:

\begin{enumerate}
  \item It generates the random Barabasi-Albert graph with N vertices and average connectivity equal to Z. Both parameters are chosen deliberately at the beginning of the code.
  \item The next step is constructing an array of strategies. This array is of length 1000 (number of players). The strategies are represented by numbers 1 (cooperation) and 2 (defection). The choice of these numbers is strict because later on, we use them to index the payoffs matrix.
        The process of creating such an array is following:
        \begin{enumerate}
          \item Creating two vectors of length 500, one filled with ones, and one filled with twos.
          \item Concatenate those two vectors to get a vector of length 1000.
          \item Shuffle the values in a vector and save them as STRATEGIES.
        \end{enumerate}
  \item For the inner loop to conduct games, we need to transform our SF NOC into two arrays --- one containing edges (tuples of players) and the other containing strategies for each edge. To achieve this, we apply the following steps:
        \begin{enumerate}
          \item We use function COLLECT on the function EDGES used on the Barabasi-Albert model, which results in an iterable, however not easily callable, array of edges.  
          \item Then we map a function TRANSFORM onto the previously obtained object. This produces us an array of tuples, each tuple representing one edge.
          \item Lastly, we map function STRAT onto an array of tuples representing the edges.
        \end{enumerate}
  \item The array of accumulated payoffs is created --- initially with all values equal to zero (after each game, the results are added to the corresponding values in this array).
  \item A global variable is created. This variable is named TRACK and it is a vector of length 100. This vector is used by the inner loop to store the proportion of cooperators in the population.
  \item The final step before triggering the inner loop is creating ARRPATH, which will be used to store 100 TRACK variables. If it is the first iteration of this loop, we create a global variable named ARRPATH, which is an array. If, on the other hand, it is not the first iteration, we are using the function PUSH! to 
        add the next TRACK to our existing array. 
\end{enumerate}


\subsection{Simulations description}

Because of the nature of this paper (an attempt to clone the results of The Paper), our simulations must be conducted in strict accordance with the methods outlined
in The Paper. Therefore it is only natural that we must follow each step with the utmost care and diligence. According to The Paper, we must conduct 100 simulations 
for each parametrization. Each simulation is performed following those steps: Where the parametrization is as follows:
\begin{enumerate}
  \item Setting up the parameters which are needed to create SF NOCs (such as the number of final vertices (population size), the average connectivity, etc.).
  \item Choosing the parameters of the game which is to be simulated (either PD or SG).
  \item Creating the randomly generated SF NOC (we use Barabasi - Albert model to do that). The SF NOC must be created in compliance with preferential attachment and 
  growth rules.
  \item Randomly distributing strategies amongst the population (SF NOC in this particular case). Each vertex can either get a cooperation or deflection strategy.
  \item Each pair of cooperator-deflectors engages in a round of a given game. In compliance with replicators dynamics, we keep track of cumulative payoffs for both 
  strategies so that "players" can adjust their strategies throughout the population. This step is repeated 11 000 times, each time is called "generation". The first 
  10 000 is the so-called "transient time".
  \item We collect results (equilibrium frequencies of cooperators and defectors) by averaging over the last 1 000 generations.
\end{enumerate}

%In order to efficiently perform the necessary calculations, we had to pay close attention to the types of variables and data structures.
%Since Julia allows us to specify the variable type (for example Int16, Float64, etc.) to save the memory space. Therefore 
%we can speed up the calculations and ultimately the entire process. 

\todo{Poniżej wrzuciłem przykładowy algorytm w pseudokodzie. Myślę, że to jest to co chcemy wykorzystać aby zapisać sam algorytm. To oczywiście musi być potem opisane w tekście.}

\subsection{Compiling \LaTeX files}

\begin{algorithm}[H]
 \KwData{this text}
 \KwResult{how to write algorithm with \LaTeX2e }
 initialization\;
 \While{not at end of this document}{
  read current\;
  \eIf{understand}{
   go to next section\;
   current section becomes this one\;
   }{
   go back to the beginning of current section\;
  }
 }
 \caption{How to write algorithms}
\end{algorithm}

% --- chapter ---------------------------------------------------------
\clearpage
\section{Results and discussion}

\todo{Tutaj wpisujemy uzyskane wyniki oraz dyskutujemy je, np. porównujemy do wyników z oryginalnego artykułu.}

% --- chapter ---------------------------------------------------------
\clearpage
\section{Conclussions}

\todo{Tutaj na samym końcu dopisujemy dlaczego i co zrobiliśmy oraz jakie mamy dalsze plany na badania.}

%\begin{table}[hbt]
%  \centering
%\end{table}
%% --- chapter ---------------------------------------------------------
%\clearpage
%\section{Basic things}
%
%\subsection{Compiling \LaTeX files}
%
%The \verb+.tex+ file is just a plain text file. It contains the \LaTeX formatting codes together with the content of a paper. To get a \verb+.pdf+ file you have to compile the \verb+.tex+ file using a sequence \verb+pdflatex+, \verb+biblatex+, \verb+pdflatex+, \verb+pdflatex+. This sequence is a default in most editors designed for use with \LaTeX.
%
%\subsection{Basic formatting for a text}
%
%Paragraphs are coded by an empty line. That is is you want to start a new paragraph it is enough to leave an empty line and start typing like that:
%\begin{verbatim}
%This is the first paragraph.
%
%This is the next paragraph.
%\end{verbatim}
%
%Everything about the paragraph is formatted for you including all indents and spacings. Again, you don't have to take care of it manually.
%
%Basic text formatting, e.g. bold face and italic, is achieved with the following commands: \verb+\textbf{}+, \verb+\textit{}+, \verb+\underline{}+, producing \textbf{text}, \textit{text}, \underline{text}. I suggest not overusing those commands!
%
%Alignment is done through environments \verb+center+, \verb+flushleft+ and \verb+\flushright+ giving the following examples.
%
%\begin{center}
%  This is centered.
%\end{center}
%
%\begin{flushleft}
%  This is aligned to the left.
%\end{flushleft}
%
%\begin{flushright}
%  This is aligned to the right. 
%\end{flushright}
%
%In other environments it is possible to use \verb+\centering+ to center content of that environment (like in \verb+figure+ or \verb+table+ environments).
%
%\subsection{Fonts and fonts' sizes}
%
%You do not change fonts and fonts' sizes! Technically it can be done but I will reject this.

%% --- chapter ---------------------------------------------------------
%\clearpage
%\section{Mathematics}
%
%This is testing footnotes\footnote{This is a footnote. We can put some math here \( x^2 - f(x) = g(x^2) \) which is not encouraged but sometimes necessary. The other thing we can do is to put here an URL \url{https://tex.stackexchange.com/questions/249415/set-font-size-for-footnotes}. }.
%
%\subsection{Basic mathematics}
%
%There are two types of mathematics inside a \LaTeX{} document. The first one is the in-line mathematics and the displayed mathematics. The first one looks like this: \( F(x) = \int_{-\infty}^{x} f(\omega) d\omega \) with the code looking like this: \verb!\( F(x) = \int_{-\infty}^{x} f(\omega) d\omega \)!. The displayed mathematics looks like that
%\[
%F(x) = \int_{-\infty}^{x} f(\omega) d\omega
%\]
%with the code
%\begin{verbatim}
%\[
%F(x) = \int_{-\infty}^{x} f(\omega) d\omega
%\]
%\end{verbatim}
%As you can see the same code is formatted differently depending on the type of mathematics.
%
%\subsection{Referencing mathematics and other things}
%
%To reference mathematics (only displayed formulas) you use the \verb+equation+ environment with a \verb+\label{}+ within. The reference is done through the \verb+\ref{}+ command. The example is
%\begin{equation}
%\label{eq:this-is-very-important-equation}
%F(x) = \int_{-\infty}^{x} f(\omega) d\omega.
%\end{equation}
%To reference the equation you use the \verb+\ref{}+ command giving (\ref{eq:this-is-very-important-equation}). The \verb+\label{}+ / \verb+\ref{}+ pair works for anything that can be referenced.
%
%\subsection{Some more mathematical formulas}
%
%Here are slightly more complex formulas. Let \( A  \) be a matrix
%\[
%A =
%\left(
%\begin{bmatrix}
%1                   & \alpha^2       \\
%2                   & \sqrt{\pi} - \log(x-\sin(y))
%\end{bmatrix}^{2}
%- 
%\begin{bmatrix}
%1                   & f(x)           \\
%2                   & g(y)
%\end{bmatrix}
%\cdot
%\begin{bmatrix}
%x                                    \\
%y
%\end{bmatrix}
%\right),
%\]
%where
%\[
%f(x) = 
%\left\{
%  \begin{aligned}
%    \frac{1}{x}     & \quad \text{for \(x<-\frac{1}{2}\),} \\
%    \frac{1}{1+x^2} & \quad \text{for \(x \geq -\frac{1}{2}\)}
%  \end{aligned}
%\right.
%\]
%and
%\[
%g(y) = \sin\left(\frac{\mathrm{\mathbf{E}}(X)}{\cos(y) + \log(y)}\right), 
%\quad\text{where \( X \sim \mathrm{N}(0, \sigma)  \).}
%\]
%
%It is very easy to typeset a normal form game. Below is an example of such a game. 
%
%\begin{game}{3}{3}
%    & $L$    & $M$    & $H$    \\
%$L$ & $16,9$ & $3,13$ & $0,3$  \\
%$M$ & $21,1$ & $10,4$ & $-1,0$ \\
%$H$ & $9,0$  & $5,-4$ & $-5,-15$
%\end{game}
%
% --- chapter ---------------------------------------------------------
%\clearpage
%\section{Figures and tables}
%
%Both figures and tables use the same ideas. To insert a table you use the \verb+table+ environment. This is an example of a simple table.
%
%\begin{table}[hbt]
%  \centering
%
%  \captionsetup{margin=10pt,font=small,labelfont=bf,width=.8\textwidth}
%
%  \caption[Short name for a table]{This is an example of a table.}
%  \label{tab:exceptional-table}
%
%\vspace*{2ex}
%
%  \begin{tabular}{lccc}
%    Name        & property 1 & property 2 & property 3 \\ \hline
%    Michael     & 23         & 34         & --         \\
%    John        & 34         & --         & 28         \\
%    Mr. Niceguy & 123        & 231        & 312        \\ \hline
%  \end{tabular}
%\end{table}
%
%Table~\ref{tab:exceptional-table} is a very simple table and much more is possible.
%
%To insert a figure you need to have a figure. In the catalog there are two figures and the following is an example of the \verb+figure+ environment.
%
%\begin{figure}[hbt]
%  \centering
%
%  \begin{subfigure}[t]{0.45\textwidth}
%  \includegraphics[width=\textwidth]{./figure-1}
%  \end{subfigure}
%
%  \captionsetup{margin=10pt,font=small,labelfont=bf,width=.8\textwidth}
%
%  \caption[Short name]{This is just an example. \textit{Source:} own calculations.}\label{fig:xxx1}
%\end{figure}
%
%\begin{figure}[hbt]
%  \centering
%  \begin{subfigure}[t]{0.45\textwidth}
%    \includegraphics[width=\textwidth]{./figure-1}
%    \caption{This is a caption for the first figure. This caption is wrapped at the right width and the hight is being compensated.}
%    \label{fig:xxxa}
%  \end{subfigure}
%  \hfill
%  \begin{subfigure}[t]{0.45\textwidth}
%    \includegraphics[width=\textwidth]{figure-2}
%    \caption{This is another caption.}
%    \label{fig:xxxb}
%  \end{subfigure}
%  
%  \captionsetup{margin=10pt,font=small,labelfont=bf,width=.8\textwidth}
%
%  \caption[Short caption 2]{This is the main caption and it is below the figures. \textit{Source:} own calculations}\label{fig:xxx}
%\end{figure}
%
%Figure~\ref{fig:xxx} is a slightly more complex than just a simple figure but it is useful to have such template. It is possible to refrence subfigures as \ref{fig:xxxa} and \ref{fig:xxxb}.

%% --- chapter ---------------------------------------------------------
%\clearpage
%\section{Bibliography}
%
%\begin{wrapfigure}{r}{.5\textwidth}
%\centering
%
%\includegraphics[width=.4\textwidth]{figure-2}
%
%\captionsetup{margin=10pt,font=small,labelfont=bf,width=.42\textwidth}
%
%  \caption[Short caption 2]{This is how one can wrap a text around a figure. \textit{Source:} own calculations}\label{fig:yyy}
%
%
%\end{wrapfigure}
%
%The content for the bibliography is in a different file named \verb+refs.bib+. You can change the name but then you have to change the information in this file from \verb+\bibliography{refs}+ to \verb+\bibliography{new-name}+ where \verb+new-name+ is the name of your file. The file \verb+refs.bib+ contains some examples for books and papers.
%
%The process of citation is simple. The command  \verb+\cite{garland2010}+ gives this \cite{garland2010} and puts all information into the bibliography section  at the end. Everything is sorted and formatted for you so that you don't have to worry about this. An example of a paper with many authors is \cite{benaim2003} or \cite{osborne1998}. 
%
%\begin{longtable}{rrrrr}
%\caption{Binary variables used in the VAR model}\label{tab:1}     \\
%  \hline
% t   & year & elections & crises & tax cuts                       \\ 
%  \hline
%  \endfirsthead
%  \multicolumn{5}{c}%
%{\tablename\ \thetable\ -- \textit{Continued from previous page}} \\
%\hline
%t    & year & elections & crises & tax cuts                       \\ 
%\hline
%\endhead
%\hline \multicolumn{5}{r}{\textit{Continued on next page}}        \\
%\endfoot
%\hline
%\endlastfoot
%  1  & 1961 & 0         & 0      & 0                              \\ 
%  2  & 1962 & 0         & 0      & 0                              \\ 
%  3  & 1963 & 0         & 0      & 0                              \\ 
%  4  & 1964 & 1         & 0      & 0                              \\ 
%  5  & 1965 & 0         & 0      & 1                              \\ 
%  6  & 1966 & 0         & 0      & 0                              \\ 
%  7  & 1967 & 0         & 0      & 0                              \\ 
%  8  & 1968 & 1         & 0      & 0                              \\ 
%  9  & 1969 & 0         & 0      & 0                              \\ 
%  10 & 1970 & 0         & 0      & 0                              \\ 
%  11 & 1971 & 0         & 0      & 0                              \\ 
%  12 & 1972 & 1         & 0      & 0                              \\ 
%  13 & 1973 & 0         & 0      & 0                              \\ 
%  14 & 1974 & 0         & 1      & 0                              \\ 
%  15 & 1975 & 0         & 1      & 0                              \\ 
%  16 & 1976 & 1         & 0      & 0                              \\ 
%  17 & 1977 & 0         & 0      & 0                              \\ 
%  18 & 1978 & 0         & 0      & 0                              \\ 
%  19 & 1979 & 0         & 0      & 0                              \\ 
%  20 & 1980 & 1         & 0      & 0                              \\ 
%  21 & 1981 & 0         & 0      & 0                              \\ 
%  22 & 1982 & 0         & 1      & 1                              \\ 
%  23 & 1983 & 0         & 0      & 0                              \\ 
%  24 & 1984 & 1         & 0      & 0                              \\ 
%  25 & 1985 & 0         & 0      & 0                              \\ 
%  26 & 1986 & 0         & 0      & 1                              \\ 
%  27 & 1987 & 0         & 0      & 0                              \\ 
%  28 & 1988 & 1         & 0      & 0                              \\ 
%  29 & 1989 & 0         & 0      & 0                              \\ 
%  30 & 1990 & 0         & 0      & 0                              \\ 
%  31 & 1991 & 0         & 1      & 0                              \\ 
%  32 & 1992 & 1         & 0      & 0                              \\ 
%  33 & 1993 & 0         & 0      & 0                              \\ 
%  34 & 1994 & 0         & 0      & 0                              \\ 
%  35 & 1995 & 0         & 0      & 0                              \\ 
%  36 & 1996 & 1         & 0      & 0                              \\ 
%  37 & 1997 & 0         & 0      & 0                              \\ 
%  38 & 1998 & 0         & 0      & 0                              \\ 
%  39 & 1999 & 0         & 0      & 0                              \\ 
%  40 & 2000 & 1         & 0      & 0                              \\ 
%  41 & 2001 & 0         & 1      & 1                              \\ 
%  42 & 2002 & 0         & 0      & 1                              \\ 
%  43 & 2003 & 0         & 0      & 1                              \\ 
%  44 & 2004 & 1         & 0      & 0                              \\ 
%  45 & 2005 & 0         & 0      & 0                              \\ 
%  46 & 2006 & 0         & 0      & 0                              \\ 
%  47 & 2007 & 0         & 0      & 0                              \\ 
%  48 & 2008 & 1         & 1      & 0                              \\ 
%  49 & 2009 & 0         & 1      & 1                              \\ 
%  50 & 2010 & 0         & 0      & 1                              \\ 
%  51 & 2011 & 0         & 0      & 0                              \\ 
%  52 & 2012 & 1         & 0      & 0                              \\ 
%  53 & 2013 & 0         & 0      & 0                              \\ 
%  54 & 2014 & 0         & 0      & 0                              \\ 
%  55 & 2015 & 0         & 0      & 0                              \\ 
%   \hline
%\end{longtable}
%

%\includegraphics[width=.4\textwidth]{figure-2}
%
%\captionsetup{margin=10pt,font=small,labelfont=bf,width=.42\textwidth}

%% --- appendices ------------------------------------------------------
%\appendix
%
%% ---------------------------------------------------------------------
%\clearpage
%\section{Appendix: Some important stuff}
%
%This appendix contains all the necessary important stuff, blah, blah, blah ...
%
%\begin{landscape}
%{\footnotesize
%\begin{longtable}{lll}
%\caption{Tutaj jest tytuł tablicy}\label{tab:nowatablica1}\\
%\hline
%Nazwa atrybutu & Wartości & Opis \\ 
%\hline
%\endfirsthead
%\multicolumn{3}{c}%
%{\tablename\ \thetable\ -- \textit{kontynuacja z poprzedniej strony}} \\
%\hline
%Nazwa atrybutu & Wartości & Opis \\
%\hline
%\endhead
%\hline \multicolumn{3}{r}{\textit{kontynuowane na następnej stronie}} \\
%\endfoot
%\hline
%\endlastfoot
%chk\_acct & - & stan środków na rachunku bieżącym (jakościowa)\\ 
% & A11 & ... \textless 0 Marek Niemieckich\\  
% & A12 & 0 \textless ... \textless 200 Marek Niemieckich\\  
% & A13 & ... \textgreater 200 Marek Niemieckich\\  
% & A14 & brak rachunku bieżącego\\  
%duration & - & czas trwania kredytu w miesiącach (numeryczna)\\  
%history & - & przeszłość kredytowa (jakościowa)\\  
% & A30 & brak kredytów w historii/wszystkie kredyty poprawnie spłacone\\  
% & A31 & wszystkie kredyty poprawnie spłacone (zaciągnięte w tym banku)\\  
% & A32 & kredyty poprawnie spłacane po dzień dzisiejszy\\  
% & A33 & opóźnienia w poprzednich spłatach kredytu\\  
% & A34 & konto krytyczne/zaciągnięte kredyty w innych bankach\\  
%purpose & - & cel (jakościowa)\\  
% & A40 & nowy samochód\\  
% & A41 & używany samochód\\  
% & A42 & meble\\  
% & A43 & telewizor\\  
% & A44 & urządzenia gospodarstwa domowego\\  
% & A45 & remont\\  
% & A46 & edukacja\\  
% & A47 & wakacje\\  
% & A48 & przekwalifikowanie\\  
% & A49 & biznes\\  
% & A410 & inne\\  
%amount & - & kwota kredytu (numeryczna)\\  
%say\_acct & - & saldo na rachunku oszczędnościowym/wartość posiadanych obligacji (jakościowa)\\  
% & A61 & ... \textless100 Marek Niemieckich\\  
% & A62 & 100 \textless= ... \textless 500 Marek Niemieckich\\  
% & A63 & 500 \textless= ... \textless 1000 Marek Niemieckich\\  
% & A64 & ... \textgreater= 1000 Marek Niemieckich\\  
% & A65 & nieznane/ brak oszczędności\\  
%employment & - & czas zatrudnienia w obecnej pracy (jakościowa)\\  
% & A71 & brak zatrudnienia\\  
% & A72 & ... \textless 1 rok\\  
% & A73 & 1 \textless= ... \textless 4 lata\\  
% & A74 & 4 \textless= ... \textless 7 lat\\  
% & A75 & ... \textgreater= 7 lat\\  
%install\_rate & - & wielkość raty jako procent rozporządzalnego przychodu (liczbowa)\\  
%pstatus & - & płeć i stan cywilny (jakościowa)\\  
% & A91 & mężczyzna; rozwodnik/w separacji\\  
% & A92 & kobieta; rozwiedziona/ w separacji/ mężatka\\  
% & A93 & mężczyzna ; wolny\\  
% & A94 & mężczyzna ; żonaty/ wdowiec\\  
% & A95 & kobieta ; wolna\\  
%other\_debtor & - & inni dłużnicy/ poręczyciele (jakościowa)\\  
% & A101 & brak\\  
% & A102 & współkredytobiorca\\  
% & A103 & poręczyciel\\  
%property & - & własność/ mienie (jakościowa)\\  
% & A121 & nieruchomość\\  
% & A122 & (jeśli nie A121) umowa oszczędnościowa/ ubezpieczenie na życie\\  
% & A123 & (jeśli nie A121/A122) samochód lub inne\\  
% & A124 & nieznane\\  
%timer\_resid & - & czas zamieszkania w aktualnym miejscu zamieszkania (liczbowa)\\  
%age & - & wiek w latach (liczbowa)\\  
%other\_install & - & inne zobowiązania ratalne (jakościowa)\\  
% & A141 & bank\\  
% & A142 & sklepy\\  
% & A143 & brak\\  
%housing & - & warunki mieszkaniowe (jakościowa)\\  
% & A151 & wynajem\\  
% & A152 & własność\\  
% & A153 & zamieszkanie bez ponoszenia kosztów\\  
%other\_credits & - & liczba aktualnych kredytów w tym banku (liczbowa)\\  
%job & - & praca (jakościowa)\\  
% & A171 & bezrobotny/niewykwalifikowany; cudzoziemiec\\  
% & A172 & niewykwalifikowany; rezydent\\  
% & A173 & wykwalifikowany pracownik/urzędnik\\  
% & A174 & menadżer/ samozatrudniony/ wysocewykwalifikowany/ wyższy urzędnik\\  
%num\_depend & - & liczba osób na utrzymaniu (liczbowa)\\  
%telephone & - & telefon (jakościowa)\\  
% & A191 & brak\\  
% & A192 & tak, zarejestrowany pod nazwiskiem klienta\\  
%foreign & - & pracownik zagraniczny (jakościowa)\\  
% & A201 & tak\\  
% & A202 & nie\\  
%response & - & decyzja kredytowa\\  
% & 1 & tak\\  
% & 2 & nie\\ 
% \hline
%\end{longtable}}
%\end{landscape}
%


% --- bibliography ----------------------------------------------------
\clearpage
\bibliographystyle{agsm}
\bibliography{refs}

% --- abstract --------------------------------------------------------
\clearpage
\addcontentsline{toc}{section}{List of tables}
\listoftables

% --- abstract --------------------------------------------------------
\clearpage
\addcontentsline{toc}{section}{List of figures}
\listoffigures



% --- abstract --------------------------------------------------------
\clearpage
\addcontentsline{toc}{section}{Streszczenie}
\section*{Streszczenie}

Tutaj zamieszczają Państwo streszczenie pracy. Streszczenie powinno być długości około pół strony.


\end{document}


%%% Local Variables:
%%% mode: latex
%%% TeX-master: t
%%% End:
